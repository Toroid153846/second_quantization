\documentclass{ltjsarticle}


%ソースコード
\usepackage{listings}
\lstset{
  numbers=left,
  basicstyle=\ttfamily,
}
%装飾
\usepackage{color}
% 数式
\usepackage{amsmath,amssymb}
\usepackage{bm}
\usepackage{physics}
\usepackage{comment}
\usepackage{autobreak}
\usepackage{mathtools}
\usepackage{mathcommand}
\mathtoolsset{showonlyrefs=true}
% 数式処理
\usepackage{luacas}
% 画像
\usepackage{graphicx}
\usepackage{here}
\usepackage{tikz}

\title{note\_ver4}
\author{Toroid153846}
\date{\today}

\begin{document}
\maketitle

\section{第二量子化}

  \begin{align}
    \ket{\bm{r}_1\sigma_1,\dots,\bm{r}_N\sigma_N}&=\frac{1}{\sqrt{N!}}\hat{\psi}_{\sigma_1}^\dag (\bm{r}_1)\cdots\hat{\psi}_{\sigma_N}^\dag (\bm{r}_N)\ket{\rm{vac}} \\
    \hat{\psi}_\sigma (\bm{r})\ket{\rm{vac}}&=0\\
    \comm{\hat{\psi}_\sigma(\bm{r})}{\hat{\psi}_{\sigma'}(\bm{r}')}_s&=0\\
    \comm{\hat{\psi}^\dag_\sigma(\bm{r})}{\hat{\psi}^\dag_{\sigma'}(\bm{r}')}_s&=0\\
    \comm{\hat{\psi}_\sigma(\bm{r})}{\hat{\psi}^\dag_{\sigma'}(\bm{r}')}_s&=\delta(\bm{r}-\bm{r}')\delta_{\sigma,\sigma'}
  \end{align}
を前提として。
\begin{align}
  \braket{\bm{r}_1\sigma_1,\dots,\bm{r}_N\sigma_N}{\bm{r}'_1\sigma'_1,\dots,\bm{r}'_{N'}\sigma'_{N'}}=\frac{\delta_{N,N'}}{N!}\sum_P s^P\delta(\bm{r}_1-\bm{r}'_{p_1})\delta_{\sigma_1,\sigma'_{p_1}}\cdots\delta(\bm{r}_N-\bm{r}'_{p_N})\delta_{\sigma_N,\sigma'_{p_N}}
\end{align}
を示そう。\\
まずは$N=N'$の場合について考える。\\
$P(i_1,\dots,i_K)$は$1,\dots,N$が$i_1,\dots,i_K,1,\dots,\underbrace{}_{i_1,\dots,i_K},\dots,N$になるように並べ替えたときに偶置換で入れ替える場合は0,奇置換で入れ替わる場合は1とした値とする。
\begin{align}
  \braket{\bm{r}_1\sigma_1,\dots,\bm{r}_N\sigma_N}{\bm{r}'_1\sigma'_1,\dots,\bm{r}'_{N}\sigma'_{N}}
  =&\frac{1}{N!} \bra{\rm{vac}}\hat{\psi}_{\sigma_N}(\bm{r}_N)\cdots\hat{\psi}_{\sigma_1}(\bm{r}_1)\hat{\psi}_{\sigma_1}^\dag (\bm{r}'_1)\cdots\hat{\psi}_{\sigma_N}^\dag (\bm{r}'_N)\ket{\rm{vac}}\\
  =&\frac{1}{N!}\bra{\rm{vac}}\hat{\psi}_{\sigma_N}(\bm{r}_N)\cdots\hat{\psi}_{\sigma_2}(\bm{r}_2)\delta(\bm{r}_1-\bm{r}'_{1})\delta_{\sigma_1,\sigma'_{1}}\hat{\psi}_{\sigma_2}^\dag (\bm{r}'_2)\cdots\hat{\psi}_{\sigma_N}^\dag (\bm{r}'_N)\ket{\rm{vac}}\\
  &+\frac{1}{N!}s\bra{\rm{vac}}\hat{\psi}_{\sigma_N}(\bm{r}_N)\cdots\hat{\psi}_{\sigma_2}(\bm{r}_2)\hat{\psi}_{\sigma_1}^\dag (\bm{r}'_1)\hat{\psi}_{\sigma_1}(\bm{r}_1)\hat{\psi}_{\sigma_2}^\dag (\bm{r}'_2)\cdots\hat{\psi}_{\sigma_N}^\dag (\bm{r}'_N)\ket{\rm{vac}}\\
  =&\frac{1}{N!}\bra{\rm{vac}}\hat{\psi}_{\sigma_N}(\bm{r}_N)\cdots\hat{\psi}_{\sigma_2}(\bm{r}_2)\delta(\bm{r}_1-\bm{r}'_{1})\delta_{\sigma_1,\sigma'_{1}}\hat{\psi}_{\sigma_2}^\dag (\bm{r}'_2)\cdots\hat{\psi}_{\sigma_N}^\dag (\bm{r}'_N)\ket{\rm{vac}}\\
  &+\frac{1}{N!}s\bra{\rm{vac}}\hat{\psi}_{\sigma_N}(\bm{r}_N)\cdots\hat{\psi}_{\sigma_2}(\bm{r}_2)\hat{\psi}_{\sigma_1}^\dag (\bm{r}'_1)\delta(\bm{r}_1-\bm{r}'_{2})\delta_{\sigma_1,\sigma'_{2}}\hat{\psi}_{\sigma_3}^\dag (\bm{r}'_3)\cdots\hat{\psi}_{\sigma_N}^\dag (\bm{r}'_N)\ket{\rm{vac}}\\
  &+\frac{1}{N!}s^2\bra{\rm{vac}}\hat{\psi}_{\sigma_N}(\bm{r}_N)\cdots\hat{\psi}_{\sigma_2}(\bm{r}_2)\hat{\psi}_{\sigma_1}^\dag (\bm{r}'_1)\hat{\psi}_{\sigma_2}^\dag (\bm{r}'_2)\hat{\psi}_{\sigma_1}(\bm{r}_1)\hat{\psi}^\dag_{\sigma_3}(\bm{r}_3)\cdots\hat{\psi}_{\sigma_N}^\dag (\bm{r}'_N)\ket{\rm{vac}}\\
  =&\frac{1}{N!}\sum_{i_1}s^{P(i_1)}\delta(\bm{r}_1-\bm{r}'_{i_1})\delta_{\sigma_1,\sigma'_{i_1}}\\
  &\times\bra{\rm{vac}}\hat{\psi}_{\sigma_N}(\bm{r}_N)\cdots\hat{\psi}_{\sigma_2}(\bm{r}_2)\hat{\psi}_{\sigma_1}^\dag (\bm{r}'_1)\cdots\underbrace{}_{i_1}\cdots\hat{\psi}_{\sigma_N}^\dag (\bm{r}'_N)\ket{\rm{vac}}\\
  =&\frac{1}{N!}\sum_{i_1\neq i_2}s^{P(i_1,i_2)}\delta(\bm{r}_1-\bm{r}'_{i_1})\delta_{\sigma_1,\sigma'_{i_1}}\delta(\bm{r}_2-\bm{r}'_{i_2})\delta_{\sigma_2,\sigma'_{i_2}}\\
  &\times\bra{\rm{vac}}\hat{\psi}_{\sigma_N}(\bm{r}_N)\cdots\hat{\psi}_{\sigma_3}(\bm{r}_3)\hat{\psi}_{\sigma_1}^\dag (\bm{r}'_1)\cdots\underbrace{}_{i_1,i_2}\cdots\hat{\psi}_{\sigma_N}^\dag (\bm{r}'_N)\ket{\rm{vac}}\\
  =&\frac{1}{N!}\sum_P s^P\delta(\bm{r}_1-\bm{r}'_{p_1})\delta_{\sigma_1,\sigma'_{p_1}}\cdots\delta(\bm{r}_N-\bm{r}'_{p_N})\delta_{\sigma_N,\sigma'_{p_N}}\braket{\rm{vac}}{\rm{vac}}\\
  =&\frac{1}{N!}\sum_P s^P\delta(\bm{r}_1-\bm{r}'_{p_1})\delta_{\sigma_1,\sigma'_{p_1}}\cdots\delta(\bm{r}_N-\bm{r}'_{p_N})\delta_{\sigma_N,\sigma'_{p_N}}
\end{align}
てか、交換子の法則を用いればもっと分かりやすく求まりそう。
\begin{align}
  \comm{\hat{A}}{\hat{B}_1\cdots\hat{B}_n}_{s^n}=\sum_{i=1}^{n}s^{i-1}\hat{B}_1\cdots\hat{B}_{i-1}\comm{\hat{A}}{\hat{B}_i}_s\hat{B}_{i+1}\cdots\hat{B}_n
\end{align}
が成り立つので示す。
\begin{align}
  \sum_{i=1}^{n}s^{i-1}\hat{B}_1\cdots\hat{B}_{i-1}\comm{\hat{A}}{\hat{B}_i}_s\hat{B}_{i+1}\cdots\hat{B}_n
  =&\sum_{i=1}^{n}s^{i-1}\hat{B}_1\cdots\hat{B}_{i-1}\hat{A}\hat{B}_i\cdots\hat{B}_n-\sum_{i=1}^{n}s^{i}\hat{B}_1\cdots\hat{B}_i\hat{A}\hat{B}_{i+1}\cdots\hat{B}_n\\
  =&\sum_{i=1}^{n}s^{i-1}\hat{B}_1\cdots\hat{B}_{i-1}\hat{A}\hat{B}_i\cdots\hat{B}_n-\sum_{i=2}^{n+1}s^{i-1}\hat{B}_1\cdots\hat{B}_{i-1}\hat{A}\hat{B}_{i}\cdots\hat{B}_n\\
  =&\hat{A}\hat{B}_1\cdots\hat{B}_n-s^n\hat{B}_1\cdots\hat{B}_n\hat{A}\\
  =&\comm{\hat{A}}{\hat{B}_1\cdots\hat{B}_n}_{s^n}
\end{align}
これを用いて
\begin{align}
  \braket{\bm{r}_1\sigma_1,\dots,\bm{r}_N\sigma_N}{\bm{r}'_1\sigma'_1,\dots,\bm{r}'_{N}\sigma'_{N}}
  =&\frac{1}{N!} \bra{\rm{vac}}\hat{\psi}_{\sigma_N}(\bm{r}_N)\cdots\hat{\psi}_{\sigma_1}(\bm{r}_1)\hat{\psi}_{\sigma_1}^\dag (\bm{r}'_1)\cdots\hat{\psi}_{\sigma_N}^\dag (\bm{r}'_N)\ket{\rm{vac}}\\
  =&\frac{1}{N!}\bra{\rm{vac}}\hat{\psi}_{\sigma_N}(\bm{r}_N)\cdots\hat{\psi}_{\sigma_1}(\bm{r}_1)\hat{\psi}_{\sigma_1}^\dag (\bm{r}'_1)\cdots\hat{\psi}_{\sigma_N}^\dag (\bm{r}'_N)\ket{\rm{vac}}\\
  &-\frac{1}{N!}s^n\bra{\rm{vac}}\hat{\psi}_{\sigma_N}(\bm{r}_N)\cdots\hat{\psi}_{\sigma_2}(\bm{r}_2)\hat{\psi}_{\sigma_1}^\dag (\bm{r}'_1)\hat{\psi}_{\sigma_2}^\dag (\bm{r}'_2)\cdots\hat{\psi}_{\sigma_N}^\dag (\bm{r}'_N)\hat{\psi}_{\sigma_1}(\bm{r}_1)\ket{\rm{vac}}\\
  =&\frac{1}{N!}\bra{\rm{vac}}\hat{\psi}_{\sigma_N}(\bm{r}_N)\cdots\hat{\psi}_{\sigma_2}(\bm{r}_2)\comm{\hat{\psi}_{\sigma_1}(\bm{r}_1)}{\hat{\psi}_{\sigma_1}^\dag (\bm{r}'_1)\cdots\hat{\psi}_{\sigma_N}^\dag (\bm{r}'_N)}_{s^n}\ket{\rm{vac}}\\
  =&\frac{1}{N!}\sum_{i_1}s^{i_1-1}\bra{\rm{vac}}\hat{\psi}_{\sigma_N}(\bm{r}_N)\cdots\hat{\psi}_{\sigma_2}(\bm{r}_2)\\
  &\times\hat{\psi}_{\sigma_1}^\dag (\bm{r}'_1)\cdots\hat{\psi}_{\sigma_{i_1-1}}^\dag(\bm{r}_{i_1-1})\comm{\hat{\psi}_{\sigma_1}(\bm{r}_1)}{\hat{\psi}_{\sigma_{i_1}}^\dag(\bm{r}_{i_1})}_s\hat{\psi}_{\sigma_{i_1+1}}^\dag(\bm{r}_{i_1+1})\cdots\hat{\psi}_{\sigma_N}^\dag (\bm{r}'_N)\ket{\rm{vac}}\\
  =&\frac{1}{N!}\sum_{i_1}s^{P(i_1)}\delta(\bm{r}_1-\bm{r}'_{i_1})\delta_{\sigma_1,\sigma'_{i_1}}\\
  &\times\bra{\rm{vac}}\hat{\psi}_{\sigma_N}(\bm{r}_N)\cdots\hat{\psi}_{\sigma_2}(\bm{r}_2)\hat{\psi}_{\sigma_1}^\dag (\bm{r}'_1)\cdots\underbrace{}_{i_1}\cdots\hat{\psi}_{\sigma_N}^\dag (\bm{r}'_N)\ket{\rm{vac}}\\
  =&\frac{1}{N!}\sum_{i_1\neq i_2}s^{P(i_1,i_2)}\delta(\bm{r}_1-\bm{r}'_{i_1})\delta_{\sigma_1,\sigma'_{i_1}}\delta(\bm{r}_2-\bm{r}'_{i_2})\delta_{\sigma_2,\sigma'_{i_2}}\\
  &\times\bra{\rm{vac}}\hat{\psi}_{\sigma_N}(\bm{r}_N)\cdots\hat{\psi}_{\sigma_3}(\bm{r}_3)\hat{\psi}_{\sigma_1}^\dag (\bm{r}'_1)\cdots\underbrace{}_{i_1,i_2}\cdots\hat{\psi}_{\sigma_N}^\dag (\bm{r}'_N)\ket{\rm{vac}}\\
  =&\frac{1}{N!}\sum_P s^{P(i_1,\dots,i_N)}\delta(\bm{r}_1-\bm{r}'_{p_1})\delta_{\sigma_1,\sigma'_{p_1}}\cdots\delta(\bm{r}_N-\bm{r}'_{p_N})\delta_{\sigma_N,\sigma'_{p_N}}\braket{\rm{vac}}{\rm{vac}}\\
  =&\frac{1}{N!}\sum_P s^P\delta(\bm{r}_1-\bm{r}'_{p_1})\delta_{\sigma_1,\sigma'_{p_1}}\cdots\delta(\bm{r}_N-\bm{r}'_{p_N})\delta_{\sigma_N,\sigma'_{p_N}}
\end{align}
交換子を用いると、計算自体は追いやすいが、なぜ$s^P$になるかを追いにくくなる。\\
$N<N'$については
\begin{align}
  \braket{\bm{r}_1\sigma_1,\dots,\bm{r}_N\sigma_N}{\bm{r}'_1\sigma'_1,\dots,\bm{r}'_{N'}\sigma'_{N'}}
  =&\frac{1}{N!}\sum_P s^P\delta(\bm{r}_1-\bm{r}'_{p_1})\delta_{\sigma_1,\sigma'_{p_1}}\cdots\delta(\bm{r}_N-\bm{r}'_{p_N})\delta_{\sigma_N,\sigma'_{p_N}}\\
  &\times\bra{\rm{vac}}\hat{\psi}_{\sigma_1}^\dag (\bm{r}'_1)\cdots\underbrace{}_{i_1,\dots,i_N}\cdots\hat{\psi}_{\sigma_{N'}}^\dag (\bm{r}'_{N'})\ket{\rm{vac}}\\
  =&0
\end{align}
$N>N'$も同様にできる。\\
\section{摂動論}
\subsection{縮退のある摂動論}
\subsubsection{前提条件}
非摂動状態のハミルトニアン$\hat{H}_0$とその直交規格化された固有状態$\ket*{\phi^{(0)}_{n,\alpha}}$と固有値$E^{(0)}_{n}$が与えられているとする。つまり、
\begin{align}
  \hat{H}_0\ket*{\phi^{(0)}_{n,\alpha}}=E^{(0)}_{n}\ket*{\phi^{(0)}_{n,\alpha}}\\
  \braket*{\phi^{(0)}_{n,\alpha}}{\phi^{(0)}_{n,\alpha}}=\delta_{\alpha,\alpha'}
\end{align}
を満たす。\\
摂動状態のハミルトニアン$\hat{H}=\hat{H}_0+\lambda\hat{V}$とその固有状態$\ket*{\varphi_{n,\alpha}}$と固有値$E_{n,\alpha}$とする。つまり、
\begin{align}
  \hat{H}\ket*{\varphi_{n,\alpha}}=E_{n,\alpha}\ket*{\varphi_{n,\alpha}}
  \label{eq:perturbation_schrodinger_equation}
\end{align}
を満たす。\\
\subsubsection{摂動論の級数展開}
\begin{align}
  E_{n,\alpha}=&E_n^{(0)}+\sum_{k=1}^{\infty}\lambda^kE^{(k)}_{n,\alpha}\\
  \ket*{\varphi_{n,\alpha}}=&\sum_{k=0}^{\infty}\lambda^k\ket*{\varphi^{(k)}_{n,\alpha}}
\end{align}
\subsubsection{射影演算子の定義と性質}
エネルギー固有状態$E^{(0)}_{n}$の縮退した固有状態が張る部分空間に射影する射影演算子$\hat{P}:=\sum^{N_n}_\beta\ketbra*{\phi^{(0)}_{n,\beta}}{\phi^{(0)}_{n,\beta}}$とそれ以外に射影する射影演算子$\hat{Q}:=\sum_{m(\neq n)}\sum^{N_m}_\beta\ketbra*{\phi^{(0)}_{m,\beta}}{\phi^{(0)}_{m,\beta}}=1-\hat{P}$を定義しておく。\\
ここで、$\hat{P}\hat{H}_0\hat{Q}=\hat{Q}\hat{H}_0\hat{P}=0$と$\hat{P}\hat{P}=\hat{P},\hat{Q}\hat{Q}=\hat{Q}$を示す。\\
任意の$\bra{\psi},\ket{\psi'}$について、$\bra{\psi}\hat{P}$はエネルギー固有状態$E^{(0)}_{n}$の縮退した固有状態が張る部分空間における状態であり、$\hat{H}_0\hat{Q}\ket{\psi'}$はエネルギー固有状態$E^{(0)}_{m}$の縮退した固有状態が張る部分空間以外における状態である。\\
したがって、
\begin{align}
  \bra{\psi}\hat{P}\hat{H}_0\hat{Q}\ket{\psi'}&=0
\end{align}
また、同様に
\begin{align}
  \bra{\psi}\hat{Q}\hat{H}_0\hat{P}\ket{\psi'}&=0
\end{align}
が成り立つ。\\
また、
\begin{align}
  \hat{P}\hat{P}=&\sum_{\beta'=1}^{N_n}\ketbra*{\phi^{(0)}_{n,\beta'}}{\phi^{(0)}_{n,\beta'}}\sum_{\beta=1}^{N_n}\ketbra*{\phi^{(0)}_{n,\beta}}{\phi^{(0)}_{n,\beta}}\\
  =&\sum_{\beta=1}^{N_n}\sum_{\beta'=1}^{N_n}\delta_{\beta,\beta'}\ketbra*{\phi^{(0)}_{n,\beta'}}{\phi^{(0)}_{n,\beta}}\\
  =&\sum_{\beta=1}^{N_n}\ketbra*{\phi^{(0)}_{n,\beta}}{\phi^{(0)}_{n,\beta}}\\
  =&\hat{P}\\
\end{align}
さらに、
\begin{align}
  \hat{Q}\hat{Q}=&(\hat{1}-\hat{P})(\hat{1}-\hat{P})\\
  =&1-2\hat{P}+\hat{P}\hat{P}\\
  =&1-2\hat{P}+\hat{P}\\
  =&1-\hat{P}\\
  =&\hat{Q}\\
\end{align}
よって、示された。\\
\subsubsection{射影演算子の摂動論への適用}
これらを用いて\eqref{eq:perturbation_schrodinger_equation}式に左から$\hat{P}$をかけると、
\begin{align}
  \hat{P}\hat{H}\ket*{\varphi_{n,\alpha}}=&\hat{P}E_{n,\alpha}\ket*{\varphi_{n,\alpha}}\\
  \hat{P}\left( \hat{H}_0+\lambda\hat{V} \right) \left( \hat{P}+\hat{Q} \right) \ket*{\varphi_{n,\alpha}}=&E_{n,\alpha}\hat{P}\ket*{\varphi_{n,\alpha}}\\
  \hat{P}\hat{H}_0\hat{P}\ket*{\varphi_{n,\alpha}}+\hat{P}\hat{H}_0\hat{Q}\ket*{\varphi_{n,\alpha}}+\lambda\hat{P}\hat{V}\hat{P}\ket*{\varphi_{n,\alpha}}+\lambda\hat{P}\hat{V}\hat{Q}\ket*{\varphi_{n,\alpha}}=&E_{n,\alpha}\hat{P}\ket*{\varphi_{n,\alpha}}\\
  \hat{P}\hat{H}_0\hat{P}\ket*{\varphi_{n,\alpha}}+\lambda\hat{P}\hat{V}\hat{P}\ket*{\varphi_{n,\alpha}}+\lambda\hat{P}\hat{V}\hat{Q}\ket*{\varphi_{n,\alpha}}=&E_{n,\alpha}\hat{P}\ket*{\varphi_{n,\alpha}}\\
  \hat{P}\hat{H}_0\hat{P}\ket*{\varphi_{n,\alpha}}+\lambda\hat{P}\hat{V}\hat{P}\ket*{\varphi_{n,\alpha}}+\lambda\hat{P}\hat{V}\hat{Q}\ket*{\varphi_{n,\alpha}}=&E_{n,\alpha}\hat{P}\hat{P}\ket*{\varphi_{n,\alpha}}\\
  \hat{P}\hat{H}_0\hat{P}\ket*{\varphi_{n,\alpha}}+\lambda\hat{P}\hat{V}\hat{P}\ket*{\varphi_{n,\alpha}}-E_{n,\alpha}\hat{P}\hat{P}\ket*{\varphi_{n,\alpha}}=&-\lambda\hat{P}\hat{V}\hat{Q}\ket*{\varphi_{n,\alpha}}\\
  \hat{P}\left( \hat{H}_0+\lambda\hat{V}-E_{n,\alpha} \right)\hat{P}\ket*{\varphi_{n,\alpha}}=&-\hat{P}\lambda\hat{V}\hat{Q}\ket*{\varphi_{n,\alpha}}\\
  \label{eq:perturbation_p}
\end{align}
同様に\eqref{eq:perturbation_schrodinger_equation}式に左から$\hat{Q}$をかけると、
\begin{align}
  \hat{Q}\left( \hat{H}_0+\lambda\hat{V}-E_{n,\alpha} \right)\hat{Q}\ket*{\varphi_{n,\alpha}}=&-\hat{Q}\lambda\hat{V}\hat{P}\ket*{\varphi_{n,\alpha}}\\
\end{align}
が求まる。\\
この式の両辺に左から$Q$の部分空間における逆演算子$\left( \hat{H}_0+\lambda\hat{V}-E_{n,\alpha} \right)^{-1}$($\hat{H}_0+\lambda\hat{V}-E_{n,\alpha}$の$Q$の部分空間の成分による演算子のみを取り出した演算子のインバース)をかけると
\begin{align}
  \hat{Q}\ket*{\varphi_{n,\alpha}}=&-\hat{Q}\left( \hat{H}_0+\lambda \hat{V}-E_{n,\alpha} \right)^{-1}\hat{Q}\lambda\hat{V}\hat{P}\ket*{\varphi_{n,\alpha}}
  \label{eq:perturbation_q}
\end{align}
よって、この結果を\eqref{eq:perturbation_p}式に代入すると
\begin{align}
  \hat{P}\left( \hat{H}_0+\lambda\hat{V}-E_{n,\alpha} \right)\hat{P}\ket*{\varphi_{n,\alpha}}=&-\hat{P}\lambda\hat{V}\hat{Q}\ket*{\varphi_{n,\alpha}}\\
  \hat{P}\left( \hat{H}_0+\lambda\hat{V}-E_{n,\alpha} \right)\hat{P}\ket*{\varphi_{n,\alpha}}=&\hat{P}\lambda\hat{V}\hat{Q}\left( \hat{H}_0+\lambda \hat{V}-E_{n,\alpha} \right)^{-1}\hat{Q}\lambda\hat{V}\hat{P}\ket*{\varphi_{n,\alpha}}\\
  \hat{P}\left( \hat{H}_0+\lambda\hat{V}+\lambda^2\hat{V}\hat{Q}\left( E_{n,\alpha}-\hat{H}_0-\lambda \hat{V} \right)^{-1}\hat{Q}\hat{V} \right)\hat{P}\ket*{\varphi_{n,\alpha}}=&E_{n,\alpha}\hat{P}\ket*{\varphi_{n,\alpha}}
  \label{eq:perturbation_result}
\end{align}
\subsubsection{摂動論の計算}
\eqref{eq:perturbation_result}式に級数展開を代入すると
\begin{align}
  &\hat{P}\left( \hat{H}_0+\lambda\hat{V}+\lambda^2\hat{V}\hat{Q}\left( E_n^{(0)}+\sum_{k=1}^{\infty}\lambda^k E^{(k)}_{n,\alpha}-\hat{H}_0-\lambda \hat{V} \right)^{-1}\hat{Q}\hat{V} \right)\hat{P}\left( \sum_{k=0}^{\infty}\lambda^k\ket*{\varphi^{(k)}_{n,\alpha}} \right) \\
  &=\left( E_n^{(0)}+\sum_{k=1}^{\infty}\lambda^kE^{(k)}_{n,\alpha} \right) \hat{P}\left( \sum_{k=0}^{\infty}\lambda^k\ket*{\varphi^{(k)}_{n,\alpha}} \right) \\
\end{align}
これを用いて$n$次摂動を考える。\\
$\lambda^0$については
\begin{align}
  \hat{P}\hat{H}_0\hat{P}\ket*{\varphi^{(0)}_{n,\alpha}}=E_n^{(0)}\hat{P}\ket*{\varphi^{(0)}_{n,\alpha}}
  \label{eq:perturbation_0th}
\end{align}
と求まる。\\
$\lambda^1$については
\begin{align}
  \hat{P}\hat{H}_0\hat{P}\ket*{\varphi^{(1)}_{n,\alpha}}+\hat{P}\hat{V}\hat{P}\ket*{\varphi^{(0)}_{n,\alpha}}=&E_n^{(0)}\hat{P}\ket*{\varphi^{(1)}_{n,\alpha}}+E^{(1)}_{n,\alpha}\hat{P}\ket*{\varphi^{(0)}_{n,\alpha}}\\
  \hat{P}\hat{V}\hat{P}\ket*{\varphi^{(0)}_{n,\alpha}}=&E^{(1)}_{n,\alpha}\hat{P}\ket*{\varphi^{(0)}_{n,\alpha}}
  \label{eq:perturbation_1st}
\end{align}
1行目から2行目の変形は任意の$\ket{\psi}$について、$\hat{P}\ket{\psi}$はエネルギー固有値$E_n^{(0)}$の部分空間への射影であり、$\hat{H}_0\hat{P}\ket{\psi}=E_n^{(0)}\hat{P}\ket{\psi}$を満たし、左から$\hat{P}$をかけて$\hat{P}\hat{H}_0\hat{P}\ket{\psi}=E_n^{(0)}\hat{P}\ket{\psi}$が成り立つためである。\\
ここで、\eqref{eq:perturbation_0th}式より$\ket*{\varphi^{(0)}_{n,\alpha}}$はエネルギー固有値$E_n^{(0)}$の部分空間における状態である。したがって、その部分空間において正規直交完全系である$\ket*{\phi^{(0)}_{n,\alpha}}$を用いて、
\begin{align}
  \ket*{\varphi^{(0)}_{n,\alpha}}=&\sum_{\beta=1}^{N_n}\ket*{\phi^{(0)}_{n,\beta}}\braket*{\phi^{(0)}_{n,\beta}}{\varphi^{(0)}_{n,\alpha}}\\
  =&\sum_{\beta=1}^{N_n}c_{n,\beta;\alpha}\ket*{\phi^{(0)}_{n,\beta}}\\
\end{align}
と表せる。これを\eqref{eq:perturbation_1st}式に代入して
\begin{align}
  \hat{P}\hat{V}\hat{P}\left( \sum_{\beta=1}^{N_n}c_{n,\beta;\alpha}\ket*{\phi^{(0)}_{n,\beta}} \right) =&E^{(1)}_{n,\alpha}\hat{P}\left( \sum_{\beta=1}^{N_n}c_{n,\beta;\alpha}\ket*{\phi^{(0)}_{n,\beta}} \right) \\
  \sum_{\beta=1}^{N_n}c_{n,\beta;\alpha}\hat{V}\ket*{\phi^{(0)}_{n,\beta}} =&\sum_{\beta=1}^{N_n}E^{(1)}_{n,\alpha}c_{n,\beta;\alpha}\ket*{\phi^{(0)}_{n,\beta}}\\
  \sum_{\beta=1}^{N_n}c_{n,\beta;\alpha}\bra*{\phi^{(0)}_{n,\gamma}}\hat{V}\ket*{\phi^{(0)}_{n,\beta}} =&\sum_{\beta=1}^{N_n}E^{(1)}_{n,\alpha}c_{n,\beta;\alpha}\bra*{\phi^{(0)}_{n,\gamma}}\ket*{\phi^{(0)}_{n,\beta}}\\
  \sum_{\beta=1}^{N_n}\bra*{\phi^{(0)}_{n,\gamma}}\hat{V}\ket*{\phi^{(0)}_{n,\beta}}c_{n,\beta;\alpha}=&E^{(1)}_{n,\alpha}c_{n,\gamma;\alpha}\\
\end{align}
これらの式により、1次の摂動エネルギーと0次の固有状態が計算できる。\\
$\lambda^2$については
\begin{align}
  \hat{P}\hat{H}_0\hat{P}\ket*{\varphi^{(2)}_{n,\alpha}}+\hat{P}\hat{V}\hat{P}\ket*{\varphi^{(1)}_{n,\alpha}}+\hat{P}\hat{V}\hat{Q}\left( E_n^{(0)}-\hat{H}_0 \right)^{-1}\hat{Q}\hat{V}\hat{P}\ket*{\varphi^{(0)}_{n,\alpha}}=&E_n^{(0)}\hat{P}\ket*{\varphi^{(2)}_{n,\alpha}}+E^{(1)}_{n,\alpha}\hat{P}\ket*{\varphi^{(1)}_{n,\alpha}}+E^{(2)}_{n,\alpha}\hat{P}\ket*{\varphi^{(0)}_{n,\alpha}}\\
  \hat{P}\hat{V}\hat{P}\ket*{\varphi^{(1)}_{n,\alpha}}+\hat{P}\hat{V}\hat{Q}\left( E_n^{(0)}-\hat{H}_0 \right)^{-1}\hat{Q}\hat{V}\hat{P}\ket*{\varphi^{(0)}_{n,\alpha}}=&E^{(1)}_{n,\alpha}\hat{P}\ket*{\varphi^{(1)}_{n,\alpha}}+E^{(2)}_{n,\alpha}\hat{P}\ket*{\varphi^{(0)}_{n,\alpha}}\\
  \label{eq:perturbation_2nd}
\end{align}
1行目から2行目の変形は\label{eq:perturbation_1st}式と同様である。\\
\subsubsection{一次摂動で縮退がとけなかった場合}
ここからは、一次摂動において固有値がすべて縮退している場合について考える。つまり、任意の$\gamma=1,\dots,N_n$について$E_n^{(1)}:=E_{n,\gamma}^{(1)}$が成り立つとする。\\
この場合に射影演算子は
\begin{align}
  \hat{P}=&\sum_{\beta=1}^{N_n}\ketbra*{\phi^{(0)}_{n,\beta}}{\phi^{(0)}_{n,\beta}}\\
  =&\sum_{\beta=1}^{N_n}\ketbra*{\varphi^{(0)}_{n,\beta}}{\varphi^{(0)}_{n,\beta}}\\
\end{align}
が成り立つため
\begin{align}
  \hat{P}\hat{V}\hat{P}=E_n^{(1)}\hat{P}
\end{align}
が成り立つ。\\
これらを用いると\eqref{eq:perturbation_2nd}式より
\begin{align}
  \hat{P}\hat{V}\hat{Q}\left( E_n^{(0)}-\hat{H}_0 \right)^{-1}\hat{Q}\hat{V}\hat{P}\ket*{\varphi^{(0)}_{n,\alpha}}=&E^{(2)}_{n,\alpha}\hat{P}\ket*{\varphi^{(0)}_{n,\alpha}}\\
\end{align}
が成り立ち、有効ハミルトニアンが
\begin{align}
  \hat{H}_{\text{eff}}:=\hat{P}\hat{V}\hat{Q}\left( E_n^{(0)}-\hat{H}_0 \right)^{-1}\hat{Q}\hat{V}\hat{P}
\end{align}
として定義される。
\section{線形応答理論}
\subsection{久保公式の主張}
$t=0$において非摂動ハミルトニアン$\hat{\mathcal{H}}_0$による熱平衡状態にあり、$t>0$における摂動
\begin{align}
  \hat{\mathcal{H}}_{\textrm{ext}}(t):=-\hat{A}X(t)
\end{align}
が加えられた時の線形応答の範囲における物理量$\hat{B}$の期待値の変化\\
\begin{align}
  \Delta B(t):=&\ev*{\hat{B}(t)}-\ev*{\hat{B}}_{eq}\\
  =&\int^\infty_0\dd{t'}\phi_{BA}(t')X(t-t')\\
\end{align}
ここで、$\phi_{BA}(t)=\Tr\left[i\left[\hat{A},e^{-\beta \hat{\mathcal{H}}_0}/\Tr[e^{-\beta\hat{\mathcal{H}}_0}]\right]e^{i\hat{\mathcal{H}}_0 t}\hat{B}e^{-i\hat{\mathcal{H}}_0 t}\right]$である。\\
\subsection{久保公式の導出}
無摂動ハミルトニアン$\hat{\mathcal{H}}_0$のときのエネルギー固有値$E_i$の固有状態$\ket{\varphi_i(0)}$であり、ハミルトニアン$\hat{\mathcal{H}}(t)$を用いて時間発展させた状態を$\ket{\varphi_i(t)}$として、$t=0$においてカノニカル分布の密度行列演算子
\begin{align}
  \hat{\rho}(t):=&\sum_i\frac{e^{-\beta E_i}}{\Tr[e^{-\beta\hat{\mathcal{H}}}]}\ketbra{\varphi_i(t)}\\
\end{align}
密度行列演算子を用いて、$t=0$の物理量$B$の期待値
\begin{align}
  \Tr[\hat{\rho}(0)\hat{B}]=&\sum_i \ev{\hat{\rho}(0)\hat{B}}{\varphi_i(0)}\\
  =&\sum_i \ev{\sum_j\frac{e^{-\beta E_j}}{\Tr[e^{-\beta\hat{\mathcal{H}}_0}]}\ketbra{\varphi_j(0)}\hat{B}}{\varphi_i(0)}\\
  =&\sum_i \frac{e^{-\beta E_i}}{\Tr[e^{-\beta\hat{\mathcal{H}}_0}]}\ev{\hat{B}}{\varphi_i(0)}\\
  =&\ev*{\hat{B}}_{eq}
\end{align}
密度行列演算子の時間発展
\begin{align}
  \pdv{\hat{\rho}(t)}{t}=&\sum_i\frac{e^{-\beta E_i}}{\Tr[e^{-\beta\hat{\mathcal{H}}}]}\left(\pdv{t}\left(\ket{\varphi_i(t)}\right)\bra{\varphi_i(t)}+\ket{\varphi_i(t)}\pdv{t}\left(\bra{\varphi_i(t)}\right)\right)\\
  =&\sum_i\frac{e^{-\beta E_i}}{\Tr[e^{-\beta\hat{\mathcal{H}}}]}\left(\frac{\hat{\mathcal{H}}}{i}\ket{\varphi_i(t)}\bra{\varphi_i(t)}-\ket{\varphi_i(t)}\bra{\varphi_i(t)}\frac{\hat{\mathcal{H}}}{i}\right)\\
  =&\frac{1}{i}[\hat{\mathcal{H}},\hat{\rho}(t)]\\
\end{align}
相互作用表示の状態$\ket{\psi_I(t)}:=e^{i\hat{\mathcal{H}}_0t}\ket{\psi(t)}$より相互作用表示の密度行列演算子
\begin{align}
  \hat{\rho}_I(t)=e^{i\hat{\mathcal{H}}_0t}\hat{\rho}(t) e^{-i\hat{\mathcal{H}}_0t}\\
\end{align}
相互作用表示の密度行列演算子の時間発展
\begin{align}
  \pdv{\hat{\rho}_I(t)}{t}=&\pdv{t}(e^{i\hat{\mathcal{H}}_0t})\hat{\rho}(t) e^{-i\hat{\mathcal{H}}_0t}+e^{i\hat{\mathcal{H}}_0t}\pdv{t}(\hat{\rho}(t)) e^{-i\hat{\mathcal{H}}_0t}+e^{i\hat{\mathcal{H}}_0t}\hat{\rho}(t) \pdv{t}(e^{-i\hat{\mathcal{H}}_0t})\\
  =&e^{i\hat{\mathcal{H}}_0t}i\hat{\mathcal{H}}_0\hat{\rho}(t) e^{-i\hat{\mathcal{H}}_0t}+e^{i\hat{\mathcal{H}}_0t}\pdv{t}(\hat{\rho}(t)) e^{-i\hat{\mathcal{H}}_0t}+e^{i\hat{\mathcal{H}}_0t}\hat{\rho}(t) (-i\hat{\mathcal{H}}_0)e^{-i\hat{\mathcal{H}}_0t}\\
  =&e^{i\hat{\mathcal{H}}_0t}\frac{1}{i}[\hat{\mathcal{H}},\hat{\rho}(t)]e^{-i\hat{\mathcal{H}}_0t}-e^{i\hat{\mathcal{H}}_0t}\frac{1}{i}[\hat{\mathcal{H}}_0,\hat{\rho}(t)]e^{-i\hat{\mathcal{H}}_0t}\\
  =&e^{i\hat{\mathcal{H}}_0t}\frac{1}{i}[\hat{\mathcal{H}}_{\textrm{ext}}(t),\hat{\rho}(t)]e^{-i\hat{\mathcal{H}}_0t}
\end{align}
これを積分して
\begin{align}
  \hat{\rho}_I(t)=&\hat{\rho}_I(0)+\int_0^t\dd{t'}\pdv{\hat{\rho}_I(t')}{t}\\
  =&\hat{\rho}_I(0)+\int_0^t\dd{t'}e^{i\hat{\mathcal{H}}_0t'}\frac{1}{i}[\hat{\mathcal{H}}_{\textrm{ext}}(t'),\hat{\rho}(t')]e^{-i\hat{\mathcal{H}}_0t'}\\
\end{align}
Schrodinger表示に直して
\begin{align}
  \hat{\rho}(t)=&e^{-i\hat{\mathcal{H}}_0t}\hat{\rho}_I(t)e^{i\hat{\mathcal{H}}_0t}\\
  =&e^{-i\hat{\mathcal{H}}_0t}\hat{\rho}(0)e^{i\hat{\mathcal{H}}_0t}+\int_0^t\dd{t'}e^{-i\hat{\mathcal{H}}_0(t-t')}\frac{1}{i}[\hat{\mathcal{H}}_{\textrm{ext}}(t'),\hat{\rho}(t')]e^{i\hat{\mathcal{H}}_0(t-t')}\\
  =&\hat{\rho}(0)+\int_0^t\dd{t'}e^{-i\hat{\mathcal{H}}_0(t-t')}\frac{1}{i}[\hat{\mathcal{H}}_{\textrm{ext}}(t'),\hat{\rho}(t')]e^{i\hat{\mathcal{H}}_0(t-t')}\\
  =&\hat{\rho}(0)+\int_0^t\dd{t'}e^{-i\hat{\mathcal{H}}_0(t-t')}\frac{1}{i}[\hat{\mathcal{H}}_{\textrm{ext}}(t'),\hat{\rho}(0)]e^{i\hat{\mathcal{H}}_0(t-t')}+\order{X^2}\\
\end{align}
この期待値を計算して
\begin{align}
  \ev*{\hat{B}(t)}=&\Tr[\hat{\rho}(t)\hat{B}]\\
  =&\Tr[\hat{\rho}(0)\hat{B}]+\int_0^t\dd{t'}\Tr\left[e^{-i\hat{\mathcal{H}}_0(t-t')}\frac{1}{i}[\hat{\mathcal{H}}_{\textrm{ext}}(t'),\hat{\rho}(0)]e^{i\hat{\mathcal{H}}_0(t-t')}\hat{B}\right]\\
  =&\ev*{\hat{B}}_{eq}+\int_0^t\dd{t'}X(t')\Tr\left[e^{-i\hat{\mathcal{H}}_0(t-t')}i[\hat{A},\hat{\rho}(0)]e^{i\hat{\mathcal{H}}_0(t-t')}\hat{B}\right]\\
  =&\ev*{\hat{B}}_{eq}+\int_0^t\dd{t'}X(t')\Tr\left[i[\hat{A},\hat{\rho}(0)]e^{i\hat{\mathcal{H}}_0(t-t')}\hat{B}e^{-i\hat{\mathcal{H}}_0(t-t')}\right]\\
  =&\ev*{\hat{B}}_{eq}+\int_0^t\dd{t'}\phi_{BA}(t-t')X(t')\\
\end{align}
よって、
\begin{align}
  \Delta B(t)=&\int^\infty_0\dd{t'}\phi_{BA}(t')X(t-t')\\
\end{align}
が示された。
\end{document}